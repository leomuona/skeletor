\documentclass[11pt,twoside,a4paper]{article}
\usepackage[left=2cm,top=1cm,right=2cm,nohead,nofoot]{geometry}
\usepackage[utf8]{inputenc}
\usepackage{hyperref}
\usepackage{amssymb}
\usepackage{graphicx}
\begin{document}
\author{Jesse Niemistö \and Leo Muona}
\title{Game Engine Architecture: End-term project - Spring 2013 \\
       Manual}

\maketitle

\section{Building}
\label{sec:building}

Compiling has only been tested on Linux operating systems, but in theory it should work on all Major OSes. Software (and system) requirements are listed in chapter~\ref{sec:system_requirements}

Executing following commands on command line will compile the program.

\begin{itemize}
  \item cd build
  \item cmake ..
  \item make
\end{itemize}

Adding parameter -DCMAKE\_BUILD\_TYPE=Debug or -DCMAKE\_BUILD\_TYPE=Release to 'cmake ..' command will build the debug build or release build.

\section{System requirements}
\label{sec:system_requirements}

The following libraries are required of the system running or building the software.

\begin{itemize}
  \item OpenGL 1.4 or newer
  \item SDL 1.2.x
  \item cmake 2.6 or newer
\end{itemize}

OpenGL is used for rendering the graphics. Only fixed pipeline is used so it should work with as low as 1.4 version. SDL is used for cross-platform window creation and handling. Cmake is used as build system.

All these requirements are met by the linux machines in Helsinki University.

The actual hardware requirements are quite low, and it should perform quite well even on netbooks.

\section{Usage}
\label{sec:usage}

The first step of using the software is to build it. To do that please see chapter~\ref{sec:building}.

After building, in build directory there will be a binary named as skeletor. Skeletor uses relative paths for referencing its resource files, so it must be executed in the build directory.

After executing the software, the user is able to 'play'. Movement of the character is controlled with the usual WASD-keys. Camera movement is controlled with mouse by holding down the left mouse button and moving the mouse. Camera distance from the character is handled with mouse wheel.

\end{document}
